% Preamble 
\documentclass[11pt]{article}

%Add packages here 
\usepackage{hyperref} %This is for adding a hyperlink to the email address

% Set margins / dimensions 
\topmargin = 0.0in
\oddsidemargin = 0.0in
\evensidemargin = 0.0in
\textwidth = 6.5in
\headheight = 0pt
\headsep = 0pt
\textheight = 9.0in

% Begin the document 
\begin{document}

%Heading
%includes Name, address, phone number and email address
\centerline{\LARGE \textbf{Rohit Tripathy}}
\centerline{Apt.\#2, 225 South River Road, West Lafayette, IN-47906, USA}
\centerline{\textbf{Phone:}+1-(765)-476-6988}
\centerline{\textbf{Email}: \href{mailto:rtripath@purdue.edu}{rtripath@purdue.edu}}

%seperate heading with a horizontal line
\line(1, 0){470}

%Now start putting sections here 

%Summary section ; get this from LinkedIn
\section*{Summary}
I am a graduate student at the \textit{\textbf{Predictive Science Lab}} at \textbf{Purdue University}. My research deals with high dimensional Uncertainty Quantification. Learning high dimensional functions is a problem of massive importance in various areas of engineering (flow in porous media, contact mechanics in granular crystals, molecular dynamics, for instance). At the same time, it is a computationally intensive problem and the computational cost rises exponentially with an increase in the number of dimensions (\textit{curse of dimensionality}). We seek to devise probabilistic surrogate models based on Bayesian principles that reduce the computational expense of learning these high dimensional mappings and quantifies model-form uncertainties. In order to do so, we utilize \textit{Gaussian Processes (GP)}, a non-parametric kernel based regression methodology. We also seek to understand how the model form uncertainty propagates through the model (\textit{Uncertainty Propagation} problem).

%Education section
\section*{Education}


\section*{Work Experience}

\section*{Technical Interests}
\section*{Skills}
\section*{Publications}
\section*{Projects}
\section*{Courses}
\section*{Volunteering}
\end{document}

